\documentclass[a4paper,10pt]{article}
%\usepackage[utf8x]{inputenc}
\usepackage{listings}
\usepackage{color}
\newif\ifshowmatlab
\showmatlabtrue


%opening
\title{OpenDA course and exercises}
\author{Nils van Velzen, Martin Verlaan, Stef Hummel}

\begin{document}
\lstset{ %
 basicstyle=\ttfamily,            % the size of the fonts that are used for the code
 breakatwhitespace=false,         % sets if automatic breaks should only happen at whitespace
 breaklines=true,                 % sets automatic line breaking
 captionpos=b,                    % sets the caption-position to bottom
 columns=flexible, 		    			     	% Prevent additional spaces to be entered by the listing, when keepspaces = true --> enable copy - paste
 escapeinside={\%*}{*)},          % if you want to add LaTeX within your code
 extendedchars=true,              % lets you use non-ASCII characters; for 8-bits encodings only, does not work with UTF-8
 frame=single,	                   % adds a frame around the code
 keepspaces=true,                 % keeps spaces in text, useful for keeping indentation of code (possibly needs columns=flexible)
 numbers=none,                    % where to put the line-numbers; possible values are (none, left, right)
 numbersep=5pt,                   % how far the line-numbers are from the code
 numberstyle=\tiny\color{mygray}, % the style that is used for the line-numbers
 showspaces=false,                % show spaces everywhere adding particular underscores; it overrides 'showstringspaces'
 showstringspaces=false,          % underline spaces within strings only
 showtabs=false,                  % show tabs within strings adding particular underscores
 tabsize=2,	                      % sets default tabsize to 2 spaces
}

\maketitle

%\begin{abstract}

%\end{abstract}

\section*{Installation of OpenDA}
Before you can start with the exercises you must first install OpenDA. For the
latest instructions, you are referred to \verb|$OPENDA/doc/OpenDA_domunentation.pdf|, 
section ``Installation'' or the same document on our website \verb|www.openda.org|.

\section{Double Pendulum}
{\bf Directory:} \lstinline{exercise_double_pendulum_part1}\\

A pendulum is a rigid body that can swing under the influence of gravity. It is attached at the top so it can rotate freely in a two-dimensional plane ($x,y$).
We will assume a thin rectangular shape with the mass equally distributed. A double pendulum is a pendulum connected to the end of another pendulum. Contrary to the 
regular movement of a pendulum, the motion of a double-pendulum is very irregular when sufficient energy is put into the system. 

\begin{center}
    \includegraphics[height=4cm]{Double-compound-pendulum.png}
\end{center}

The dynamics of a double-pendulum can be described with the following equations. 
(This example was copied from \url{https://en.wikipedia.org/wiki/Double_pendulum})

With variables $\theta_1, \theta_2, p_{\theta_1}, p_{\theta_2}$:
\begin{align}
   \frac{d \theta_1}{dt}&= \frac{6}{m l^2} \frac{2 p_{\theta_1} - 3\cos(\theta_1-\theta_2) p_{\theta_2}}
   {16-9 \cos^2(\theta_1-\theta_2)}\\
   \frac{d \theta_2}{dt}&= \frac{6}{m l^2} \frac{8 p_{\theta_2} - 3\cos(\theta_1-\theta_2) p_{\theta_1}}
   {16-9 \cos^2(\theta_1-\theta_2)}\\
   \frac{dp_{\theta_1}}{dt} &= -\frac{1}{2} ml^2 \left( \frac{d \theta_1}{dt} \frac{d \theta_2}{dt} \sin(\theta_1-\theta_2) + 3\frac{g}{l} \sin(\theta_1) \right)  \\
   \frac{dp_{\theta_1}}{dt} &= -\frac{1}{2} ml^2 \left( -\frac{d \theta_1}{dt} \frac{d \theta_2}{dt} \sin(\theta_1-\theta_2) + \frac{g}{l} \sin(\theta_2) \right) 
\end{align}
where the $x,y$-position of the middle of the two segments can be computed as:
\begin{align}
   x_1 &= \frac{l}{2} \sin(\theta_1) \\
   y_1 &= \frac{-l}{2} \cos(\theta_1) \\
   x_2 &= l ( \sin(\theta_1) + \frac{1}{2} \sin(\theta_2) ) \\
   y_2 &= -l ( \cos(\theta_1) + \frac{1}{2} \cos(\theta_2) )
\end{align}

This model, although simple, is very nonlinear and has a chaotic nature.  Its
solution is very sensitive to the parameters and the initial conditions: a
small difference in those values can lead to a very different solution.

The purpose of this exercise is to get you started with OpenDA. You will learn
to run a model in OpenDA, make modifications to the input files and plot the
results.
\subsection{Input files}
The input for this exercise is located in directory \lstinline{exercise_pendulum_part1}. 

For Linux and Mac OS X, go to this directory and start \lstinline{oda_run.sh}, the
main application of OpenDA. For Windows, start the main application with 
\lstinline{oda_run_gui.bat} from the \lstinline{$OPENDA/bin} directory. The main 
application allows you to view and edit the OpenDA configuration files, run your
simulations and visualize the results.

\subsection{Simulation and postprocessing with the double pendulum model}
\subsubsection{simulation}
Try to run a simulation with the double pendulum model. You can use the configuration file \lstinline{simulation_unperturbed.oda}. 

\subsubsection{postprocessing}      
\ifshowmatlab
      For postprocessing in Matlab the results are written to the file
      \begin{center}
        \lstinline{simulation_unperturbed_results.m}
      \end{center}
      Next, start Matlab and load the results. We have added a routine \lstinline{plot_movie} to create an intuitive
      representation of the data. Please type (or copy-paste):
      \begin{lstlisting}[style=MatlabStyle,caption={Matlab}]
[t,unperturbed,tobs,obs]= ...
load_results('simulation_unperturbed_results');
plot_movie(t,unperturbed)\end{lstlisting}
\fi
      
For postprocessing in Python the results are written to the file
\begin{center}
  \texttt{ simulation\_unperturbed\_results.py}
\end{center}
      
      
These results can be loaded with:
\begin{lstlisting}[style=PythonStyle, caption={Python initialize}]
import simulation_unperturbed_results as unperturbed
# use reload(unperturbed) if unperturbed was loaded before\end{lstlisting}
      We have added a routine \lstinline{plot_movie} to create an intuitive
      representation of the data. 
\begin{lstlisting}[style=PythonStyle, caption={Python}][style=PythonStyle,frame=single,caption={Python}]
import pendulum as p #needed only once
p.plot_movie(unperturbed.model_time,unperturbed.x)\end{lstlisting}
      
\ifshowmatlab
To create a time-series plot in Matlab type:
\begin{lstlisting}[style=MatlabStyle,frame=single,caption={Matlab}]
subplot(2,1,1);
plot(t,unperturbed(1,:),'b-');
ylabel('\theta_1');
subplot(2,1,2);
plot(t,unperturbed(2,:),'b-');
ylabel('\theta_2');
xlabel('time');
\end{lstlisting}
\fi
      
To create a time-series plot in Python type:
\begin{lstlisting}[style=PythonStyle,caption={Python}]
plt.subplot(2,1,1)
plt.plot(unperturbed.model_time,unperturbed.x[:,0],"b") 
# Python counts starting at 0
plt.ylabel(r"$\theta_1$") # use latex for label
plt.subplot(2,1,2)
plt.plot(unperturbed.model_time,unperturbed.x[:,1],"b")
plt.ylabel(r"$\theta_2$")
plt.show() 
# only needed if interactive plotting is off. 
# Set with plt.ioff(), plt.ion()
\end{lstlisting}
%
\subsection{An alternative simulation with the double-pendulum model}

Then you can start an alternative simulation with the double-pendulum model that
starts with a slightly different initial condition using the
configuration file \texttt{ simulation\_perturbed.oda}. The different initial conditions
can be found in {\ttfamily model\/DoublePendulumStochModel.xml} and {\ttfamily model\/DoublePendulumStochModel\_perturbed.xml}

Visualize the unperturbed and perturbed results in a single plot. Make
       a movie and a time-series plot of $\theta_1$ and $\theta_2$ variables. Do you see
       the solutions diverging like the theory predicts?
       
\ifshowmatlab
\begin{lstlisting}[style=MatlabStyle, caption={Matlab}]
[tu,unperturbed,tobs1,obs1]=load_results('simulation_unperturbed_results');
[tp,perturbed,tobs2,obs2]=load_results('simulation_perturbed_results');
figure(1);clf;subplot(2,1,1);
plot(tu,unperturbed(1,:),'b');
hold on;
plot(tp,perturbed(1,:),'g');
hold off;
legend('unperturbed','perturbed')
subplot(2,1,2);
plot(tu,unperturbed(2,:),'b');
hold on;
plot(tp,perturbed(2,:),'g');
hold off;\end{lstlisting}
\fi
      
To create a movie with both results in python type:
\begin{lstlisting}[style=PythonStyle,caption={Python initialize}]
import simulation_unperturbed_results as unperturbed
import simulation_perturbed_results as perturbed
p.plot_movie(unperturbed.model_time, unperturbed.x, perturbed.x)
\end{lstlisting}

To create a time-series plot with both results in Python type:
\begin{lstlisting}[style=PythonStyle,caption={Python}]
plt.subplot(2,1,1)
plt.plot(unperturbed.model_time,unperturbed.x[:,0],"b") 
# Python counts starting at 0
plt.plot(perturbed.model_time,perturbed.x[:,0],"g")
plt.ylabel(r"$\theta_1$") # use LaTeX for label
plt.subplot(2,1,2)
plt.plot(unperturbed.model_time,unperturbed.x[:,1],"b")
plt.plot(perturbed.model_time,perturbed.x[:,1],"g")
plt.ylabel(r"$\theta_2$")
plt.show() \end{lstlisting}

\subsection{An ensemble of model runs}
Next, we want to create an ensemble of model runs all with slightly different initial conditions. 
      You can do this in a number of steps:
      \begin{itemize}
      \item First create the input file \texttt{ simulation\_ensemble.oda} based on\\
            \texttt{ simulation\_unperturbed.oda}. Change the algorithm and the
            configuration of the algorithm.\\
            hint: the algorithm is called \\
            org.openda.algorithms.kalmanFilter.SequentialEnsembleSimulation.
      \item Create a configuration file for the Ensemble algorithm (e.g. named\\
            \texttt{algorithm/SequentialEnsembleSimulation.xml}) with the following content:
\begin{lstlisting}[language=XML,frame=single,caption={XML-input for sequentialAlgorithm}]
<?xml version="1.0" encoding="UTF-8"?>
<sequentialAlgorithm>
  <analysisTimes type="fromObservationTimes"></analysisTimes>
  <ensembleSize>5</ensembleSize>
  <ensembleModel stochParameter="false"
                 stochForcing="false"
                 stochInit="true" />
</sequentialAlgorithm>

\end{lstlisting}
      Hint: do not forget to reference \texttt{ algorithm/SequentialEnsembleSimulation.xml} in \\ \texttt{ simulation\_ensemble.oda}
      and do not forget to give a diferent name to the output files.
      \item Run the new configuration with OpenDA.

      \item make a plot of the first and second variable of the five ensemble
      members in a single time-series plot
\ifshowmatlab
\begin{lstlisting}[style=MatlabStyle,frame=single,caption={Matlab}]
[t,ens]=load_ensemble('simulation_ensemble_results');
ens_th1=reshape(ens(1,:,:),size(ens,2),size(ens,3));
ens_th2=reshape(ens(2,:,:),size(ens,2),size(ens,3));
clf; subplot(2,1,1);
plot(t(2:end),ens_th1);
ylabel('\theta_1');
subplot(2,1,2);
plot(t(2:end),ens_th2);
ylabel('\theta_2');
xlabel('time');\end{lstlisting}
\fi
     
\begin{lstlisting}[style=PythonStyle,frame=single,caption={Python}]
import ensemble
import simulation_ensemble_results as res
(t,ens)=ensemble.reshape_ensemble(res)
ens1=ens[:,0,:] #note we start counting at 0
ens2=ens[:,1,:]
plt.subplot(2,1,1)
plt.plot(t[1:],ens1,"b")
plt.ylabel(r"$\theta_1$")
plt.subplot(2,1,2)
plt.plot(t[1:],ens2,"b")
plt.ylabel(r"$\theta_2$")
plt.show()
\end{lstlisting}
      
      \item Observations of $\theta_1$ and $\theta_2$ are available as well. Make a plot of
      the observations together with the simulation results.
\ifshowmatlab
\begin{lstlisting}[style=MatlabStyle,frame=single,caption={Matlab}]
[t,unperturbed,tobs,obs]= ...
load_results('simulation_unperturbed_results');
subplot(2,1,1);
plot(t,unperturbed(1,:),'b-');
hold on
plot(tobs,obs(1,:),'k+');
hold off
ylabel('\theta_1');
subplot(2,1,2);
plot(t,unperturbed(2,:),'b-');
hold on
plot(tobs,obs(2,:),'k+');
hold off
ylabel('\theta_2');
xlabel('time');
\end{lstlisting}
\fi
\begin{lstlisting}[style=PythonStyle,frame=single,caption={Python}]
import simulation_unperturbed_results as unperturbed
plt.subplot(2,1,1)
plt.plot(unperturbed.model_time,unperturbed.x[:,0],"b") 
plt.plot(unperturbed.analysis_time,unperturbed.obs[:,0],"k+")
plt.ylabel(r"$\theta_1$")
plt.subplot(2,1,2)
plt.plot(unperturbed.model_time,unperturbed.x[:,1],"b")
plt.plot(unperturbed.analysis_time,unperturbed.obs[:,1],"k+")
plt.ylabel(r"$\theta_2$")
plt.show()
\end{lstlisting}
      
We can see that although our simulation starts on the right track, it quickly diverges from the observations.
The aim of the Ensemble Kalman filter or data-assimilation in general, is to keep the model on track. 
\end{itemize}


\section{Some basic properties of the EnKF with Lorenz 3 variable model}
{\bf Directory: {\tt exercise\_lorenz\_3var\_part2}}\\

In this exercise you will learn how to set up and run the EnKF method in OpenDA.

\begin{itemize}
  \item Prepare the input files for a run with the EnKF method. Use the input
        files from exercise\_lorenz\_3var\_part1 as template. Hint: the Ensemble Kalman filter
        is called org.openda.algorithms.kalmanFilter.EnKF. The algorithm
        configuration file has the following content
      \begin{lstlisting}[language=XML,frame=single,caption={XML-input for EnKF algorithm}]
      <?xml version="1.0" encoding="UTF-8"?>
      <EnkfConfig>
         <ensembleSize>10</ensembleSize>
         <ensembleModel stochParameter="false"
                        stochForcing="false"
                        stochInit="true" />
      </EnkfConfig>
      \end{lstlisting}

  \item Plot the ensemble mean of the first model variable and the observations.
        With some luck the solution should track the observations.\\
        Tip: use the scripts {\tt load\_obs.m} and {\tt load\_ensemble.m} for reading the
        data into matlab (cf. Exercise1), or {\tt load\_ensemble.py} for python.
  %oda_run EnKf.oda
  %plot1.m
 \item Look at the observation input file of the StochObserver. The
       StochObserver does not only describe the observations but the accuracy
       as well. Can you make a new observation input file with similar
       observed values but with a 10 times larger standard deviation for the
       observation error.
       Tip: you can edit the file in OpenOffice or MS Excel or use the find
       and replace function of an advanced text editor.
 \item Repeat the run with EnKF but now for the new observations and plot
       the first variable of the ensemble means and the observations. What do
       you see and what is the reason for this behavior of the algorithm?
 %oda_run EnKf10.oda
 %plot1.m
 \item The number of ensemble members controls the accuracy of the ensemble
       approximation. What happens if you increase the number to e.g. 100, or
       decrease it to 5? Use (initially) observations with a standard deviation
       of 5.0. Experiment as well with various standard
       deviations of the observations.
\end{itemize}

%extra: Belang van initial seed (een enkele run is ook maar een enkele realisatie) en gebruik van "ensembles van ensembles"
%laat ze ook spelen met initiele seed van de randomgenerator (kan dat in OpenDA??) Zodat ze dan zien dat voor grote ensembles minder impact (zou moeten) hebben dan voor kleine aantallen ensembles.



\section{Steady-state Kalman Filter}
{\bf Directory: {\tt exercise\_steady\_state\_filter}}\\

  In this section you will learn how to create and use a steady-state Kalman
  filter with OpenDA. The example model we use in this section is a
  1-dimensional wave model:
  \begin{eqnarray}
  \frac{\partial h}{\partial t} + D \frac{\partial v}{\partial x} = 0 \\
  \frac{\partial v}{\partial t} + g \frac{\partial h}{\partial x} + c_f v = 0 \\
  \end{eqnarray}
  With $h(x,t)$ the (water) level above the reference plane, $v(x,t)$ the
  velocity, $D(x)$ the depth under the reference plane, $g$ the gravitational
  acceleration $c_f$ the friction coefficient and $x\in [0,L]$ the location.
  For our model we have selected the boundary values $v(x=L,t)=0$ and
  $h(x=0,t)=\frac{1}{5} \sin(2 \pi t)$. An AR(1) model is defined on the left
  water level boundary.

\begin{itemize}
\item Look at the implementation of the model in {\tt
  WaveStochModelInstance.java}, in the directory {\tt
  simple\_wave\_model/java/src/org/openda/}. See how the state is defined and how
  the model is discretized. If you want you can compile the model using {\tt ant
  build} as we will explain in excercise 6. However to make it easy for you, you
  will find the compiled version of this model,{\tt simple\_wave\_model.jar} in
  the directory {\tt simple\_wave\_model/bin}.
\item The model represents a "user" model that is not part of the OpenDA distribution. Therefore you have to copy the
 model jar-file to the bin directory of your OpenDA installation. In this way OpenDA can find this model.
\item Run the model ({\tt waveSimulation.oda}) and visualize the model results
  ({\tt plot\_movie.m} or {\tt plot\_movie.py}). Do not forget to add the jar-file of the model to the
  {\tt CLASSPATH} variable, or to copy the jar-file into the bin directory of
  your OpenDA version
\item Adjust the input files in order to run the model with stochastic
  forcings.
\item Generate water level observations from this stochastic run. We need
  observations at (approximately) $x=\frac{1}{4} L$, $x=\frac{1}{2} L$ and
  $x=\frac{3}{4} L$. You can use the script {\tt generate\_obs.m} for this
  task. We want to have observations at $t=0.1, 0.2,...,10.0$, (initially)
  select a standard deviation of 0.05.
\item Run the Ensemble Kalman filter ({\tt waveEnkf.oda}). This run will
  generate and write gain matrices at specified times. Find where and how this
  is specified in the input.
\item Plot the columns of the gain matrices. (The script {\tt plot\_gains.m}
  or {\tt plot\_gains.py}
  plots the water level part of the gain matrices). What do these columns mean?
\item (Re)generate the gain matrices using different numbers of ensembles. When
  you compare the gain matrices, what do you notice. Note: The algorithm will
  generate an enormous amount of output when you run the EnKF with a very large
  number of ensembles (e.g. 500). You can suppress the output by commenting out
  (or remove) the {\tt resultWriter}-part of the oda-input file.
\item Use the generated steady state gain matrices for a steady state Kalman
  run ({\tt waveSteadystate.oda}). Compare the performance of:
  \begin{itemize}
  \item a (non-stochastic) run without filtering,
  \item an EnkF run with various numbers of ensembles (do not forget to
    reinstate the resultWriter if you have switched it off),
  \item the various steady state gains.
  \end{itemize}
 you can use the scripts {\tt plot\_obs\_sim.m}, {\tt plot\_obs\_ens.m}
 and\\ {\tt plot\_obs\_steady.m} and similar routines for python.

\item Generate (observations) gain matrices but now for only a single
  observation. Make sure that the observed values are exactly the same as in
  the 3 observation observer. Compare the columns of the 3-observation gain
  matrices to the single observation matrices. What is the main difference and
  why?
\end{itemize}


\section{Exercise 4: A black box model - Calibration}
\newcommand{\opgave}{exercise\_4}
A simple way to connect a model to OpenDA is by letting OpenDA access the input
and output files of the model. OpenDA cannot directly understand the input and
output files of an arbitrary model. Some code has to be written such that the
black-box model implementation of OpenDA can read and write these files. In
this exercise, you will learn how to connect an existing model to OpenDA
assuming that all the input and output files of the model can indeed be
accessed by OpenDA. The exercise focuses on the configuration of the black-box
wrapper in OpenDA.

The model describes
the advection of two chemical substances. The first substance $c_1$ is emitted
as a pollutant by a number of sources. However, in this case this substance reacts
with the oxygen in the air to form a more toxic substance $c_2$. The model
implements the following equations:

\begin{eqnarray}
    \frac{\partial c_1}{\partial t} + u\frac{\partial c_1}{\partial x} & = & -
    1/T c_1, \\
    \frac{\partial c_2}{\partial t} + u\frac{\partial c_2}{\partial x} & = &
    1/T c_1.
\end{eqnarray}


In the directory 

{\tt { \opgave}/original\_model/}

you will find:
\begin{enumerate}
	\item the model executable: {\tt reactive\_pollution\_model.py} (Linux and Mac)
	      and  {\tt reactive\_pollution\_model.exe} (Windows);
	\item the model configuration file: {\tt config.yaml};
        \item the forcings of the model (injection of pollutant): {\tt forcings};
	\item the initial model state: {\tt input}.
\end{enumerate}


\begin{itemize}
	\item Run the model, in the {\tt original\_model} directory from the
	      command line, not using OpenDA.
\end{itemize}
The model generates the output files 

{\tt c1\_locA},{\tt c1\_locB}, {\tt c1\_locC}, {\tt c1\_locA}, {\tt c2\_locB}, and {\tt c2\_locC},

with time series
of substance $c_1$ and $c_2$ at three predefined locations in the model. The
folder {\tt maps} contains output files with the concentration of $c_1$ and $c_2$ on each grid point at specified times.
The folder {\tt restart} contains files that allow the model to restart; continue the computations from the point where a restart file has been written.
\begin{itemize}
	\item Investigate the input and output files of the model. 
	\item Generate a movie by running the script {\tt plot\_movie\_orig.py} script from the {\tt { \opgave}} (!) directory. This allows you  to study the behavior of the model. 
\end{itemize}

\subsection{Wrapper configuration files}

The input and output files of this model are all easy-to-interpret ASCII files. 
Therefore, we do not need model-specific routines to couple this model to OpenDA.

When you couple an arbitrary model to OpenDA and you want to use the black-box coupler, there are two approaches:
\begin{itemize}
	\item write a pre- and post-processing script that translates the (relevant)
	      model files into a more generic format that is already supported
	      (e.g. ASCII or NetCDF).
	\item write your own adapter in Java (data object) to read and write the
	      model input and output files.
\end{itemize}

A black-box wrapper configuration usually consists of three XML files. For our
pollution model, these files are:
\begin{enumerate}
   \item {\tt polluteWrapper.xml}: This file specifies how OpenDA can run the model, which input and output files are involved, and which data objects are used to interpret the model files.
     \\ This file
     consists of the parts:
     \begin{itemize}
        \item {\tt aliasDefinitions:} This is a list of strings that can be
          aliased in the other XML files. This helps to make the
          wrapper XML file more generic. E.g. the alias definition
          \verb|%outputFile%| can be used to refer to the output file of the model,
          without having to know the actual name of that output file.\\ Note
          the special alias definition \verb|%instanceNumber%|. This will be
          replaced internally at runtime with the member number of each created
          model instance.
        \item {\tt run:} the specification of what commands need to be executed
          when the model is run.
        \item {\tt inputOutput:} the list of 'input/output objects', usually
          files, that are used to access the model, i.e. to adjust the model's
          input, and to retrieve the model's results. For each 'dataObject', one
          must specify:
        \begin{itemize}
           \item the Java class that handles the reading from and/or writing to
             the file
           \item the identifier of the dataObject, so that the model
             configuration file can refer to it when specifying the model
             variables that can be accessed by OpenDA, the so-called 'exchange
             items' (see below)
           \item optionally, the arguments that are needed to initialize the
             data object, i.e. to open the file.
        \end{itemize}
     \end{itemize}
   \item {\tt polluteModel.xml}: This is the main specification of the
     (deterministic) model. It contains the following elements:
     \begin{itemize}
        \item {\tt wrapperConfig}: A reference to the wrapper config file
          mentioned above.
        \item {\tt aliasValues}: The actual values to be used for the aliases
          defined in the wrapper config file. For instance, the \verb|%configFile%|
          alias is set to the value \verb|config.yaml|.
        \item {\tt timeInfoExchangeItems}: The name of the model variables (the
          'exchange items') that can be accessed to modify the start and end
          time of the period that the model should compute to propagate
          itself to the next analysis time.
        \item {\tt exchangeItems}: The model variables that are allowed to be
          accessed by OpenDA, for instance, parameters, boundary conditions, and
          computed values at certain locations. Each variable exchange item
          consists of its id, the dataObject that contains the item, and the
          'element name', the name of the exchange item in the dataObject.
     \end{itemize}
   \item {\tt polluteStochModel.xml}: the specification of the
     stochastic model. It consists of two parts:
     \begin{itemize}
        \item {\tt modelConfig}: A reference to the deterministic model
          configuration file mentioned above {\tt polluteModel.xml}.
        \item {\tt vectorSpecification}: The specification of the vectors that
          will be accessed by the OpenDA algorithm. These vectors are grouped
          into two parts:
          \begin{itemize}
             \item The state that is manipulated by an OpenDA filtering
               algorithm, i.e. the state of the model combined with the noise
               model(s).
             \item The so-called predictions, i.e. the values on observation
               locations as computed by the model.
          \end{itemize}
     \end{itemize}
\end{enumerate}

Start with a single OpenDA run to understand where the model results appear
for this configuration:
\begin{itemize}
 \item Have a look at the files {\tt polluteWrapper.xml}, {\tt
   polluteModel.xml} and {\tt polluteStochModel.xml}, and recognize the various
   items mentioned a\-bove. Start the OpenDA GUI from the {\tt public/bin}
   directory and run the model using the {\tt Simulation.oda} configuration.
   Note that the actual model results are available in the directory where the
   black-box wrapper has let the model perform its computation: {\tt
     work/work0}.
\end{itemize}

{\bf Directory: {\tt exercise\_black\_box\_calibration\_polution}}

In this exercise, we will calibrate the value of the reaction-rate constant.
The algorithm used in this example is the Dud (which stands for Doesn't Use
Derivative).

\begin{itemize}
\item Have a look at the {\tt Dud.oda} and the configuration files it refers
  to. Run it from the OpenDA GUI and have a look at the results. What could you
  do to improve the results?

\item Figure out where to change the control parameters for the calibration
  procedure and play around with the settings to improve your results.

\end{itemize}

Calibration runs normally take longer than a few minutes. In that case, it
becomes convenient to be able to restart from a previous run.

\begin{itemize}
\item Adapt the configuration in such a way that you are able to restart the
  Dud.oda from the result of a previous run.
\end{itemize}


\section{Exercise 5: A black box model - Filtering}
%{\bf Directory: {\tt exercise\_black\_box\_enkf\_polution}}\\

\subsection{Sequential simulation}
We will first run our polution model from OpenDA using the SequentialSimulation algorithm. This run is exaclty the same as running the model outside openda. The difference is however that we provide a set of observations and run the model and restart the model between the observation times. Output will be available at the end in the generic OpenDa format that allows us to compare the model results with the available observations of the system.

\begin{itemize}
\item Run the model within OpenDA by using
	 the \\{\tt SequentialSimulation.oda} configuration. This will create the result file 
		{\tt sequentialSimulation\_results.py}. Use the script  {\tt plot\_movie\_seq.py} to visualize
		the simulation results. The script {\tt plot\_obs\_seq.py} shows the difference in time between
		the model results (prediction) and observed values of the system
\end{itemize}

\subsection{Sequential ensemble simulation}
The next step is running an ensemble of simulations. In this case we consider our main source of uncertainty the injection of polutant c1 in the model. Similar to the sequential simulation we do not assmilate any data (yet).

\begin{itemize}
 \item Run an ensemble forecast model by using the \\{\tt
   SequentialEnsembleSimulation.oda} configuration. On which variable does the
   algorithm impose stochastic forcing?\\ Have a look at the {\tt
     work} directory, and note that the black box wrapper created
   the required ensemble members by repeatedly copying the template directory
   {\tt stochModel/input} to\\ {\tt output/work\textless
     N\textgreater}.
 \item Compare the result between the mean of the ensemble and the results from {\tt
     SequentialSimulation.oda}. Note the differences. 
     You can use the script {\tt plot\_movie\_enssim.py}.
\end{itemize}

\subsection{parallel computing}
Running the ensembles takes a lot of time, especially starting the model takes quite some time compared to the actual computation time. Most computers have multiple cores and the reactive pollution model only uses one core, so we can use our cores to propage multiple ensemble members foreward in time simultaniously. 
\begin{itemize}
	\item compare the configurations {\tt SequentialEnsembleSimulation.oda} and\\
		{\tt enkf.oda} which uses parallel propagation of ensemble members. Set the number of simultanious models that corresponds to the number of cores on your computer (maxThreads).
\end{itemize}

\subsection{ensemble kalman filter}


Now let us have a look at the configuration for performing OpenDA's Ensemble
Kalman Filtering on our black box model, using a twin experiment as an example.
The model has been run with the 'real' values (time dependent)for the
concentrations for substance 1 as disposed by factory 1 and factory 2. This
'truth' stored in the directory {\tt truthmodel}, and the results of that run
have been used to generate observation time series at the output locations.
These time series (with some noise added) have been copied to the {\tt stochObserver} directory to
serve as observations for the filtering run.

The filter run takes the original unperturbed model as input, while the 'truth' 
uses a perturbed version of the original model: the concentrations for substance 1 as disposed by
factories have been flattened out to a constant value. The filter process
should modify these values in such a way that the results resemble the truth as
much as possible.

To do this the filter modifies the concentration at factory 2, and uses the
observations downstream of factory 2 to optimize the forecast.

\begin{itemize}
 \item Note that the same black box configuration is used for the sequential
   run, the sequential ensemble run, and for the EnKF run. Identify the part of
   the {\tt polluteStochModel.xml} configuration that is used only by the EnKF
   run, and not by the others.
 \item Execute the Ensemble Kalman Filtering run by using the {\tt EnKF.oda}
   configuration.\\ Check how good the run is performing, by analyzing to what
   extent the filter has adjusted the predictions towards the
   observation.\\ Note that the model output files in {\tt
     stochModel/output/work0} only contains a few time steps. Can you explain
   why?\\ So to compare the observations with the predictions you have to use
   the result file produced by the EnKF algorithm which can be visualised using
   {\tt plot\_movie\_enkf.py}.
\end{itemize}

Now let us extend the filtering process by incorporating also the concentration
disposed by factory 1, and by including the observation locations downstream of
factory 1.

\begin{itemize}
	\item Make a copy of the involved config files, {\tt EnKF.oda},\\ 
		{\tt parallel.xml}, {\tt polluteStochModel.xml} and
		{\tt timeSeriesFormatter.xml} (you could call them
		{\tt EnKF2.oda}, {\tt parallel2.xml} etc.
		\\ Adjust the files such that all references to the files are correct.
	\item Now adjust {\tt polluteStochModel2.xml} and {\tt timeSeriesFormatter2.xml}
	      in such a way that the filtering process is extended as described above.
 \item Run the filtering process by using the {\tt EnKF2.oda} configuration,
   and compare the results with the previous version of the filtering process.
\end{itemize}



\section{Exercise 6: Writing your own toy model}

{\bf Before you start:}\\
In order to be able to compile your model you need to have a (current) version
installed on your computer of:
\begin{itemize}
\item The Java Development Kit (JDK). You can download this from\\
      {\tt www.oracle.com}\footnote{Java Runtime Environment (JRE), which is
      installed on most computers is not sufficient since this will allow you
      to run java programs but it does not include the java compiler {\tt javac} that is
      needed to create you own (parts of) programs}
\item Apache Ant, this is a command line tool we use for building your java
      code. You can download Ant from {\tt ant.apache.org}.
\end{itemize}

In this exercise you will learn how to code your own model and use it in
OpenDA. The directory {\tt exercise\_6} contains a template of the code for the
1-D advection model we will create in this exercise. The content of this
directory is similar to the OpenDA directories you have seen in the previous
exercises. The difference is that we will not use a model that is already part
of the OpenDA distribution but instead our own model. The model code can be
found in the directory {\tt simple\_advection\_model}.

The model you will create is build as an extension of the OpenDA \\{\tt
  simpleStochModelInstance}. This will simplify and reduce the amount of
programming because a significant part of the implementation is already
available. For more complex models you might need to implement all methods of
the\ {\tt IStochModelInstance} class.

In the directory\\ {\tt
  exercise\_6/simple\_advection\_model/java/src/org/openda} you will find the
two java source files {\tt AdvectionStochModelFactory.java} and\\ {\tt
  AdvectionStochModelInstance.java}. The first file implements the ModelFactory
class. The model factory is a class in OpenDA that is responsible for creating
model instances (e.g. the members of an ensemble Kalman filter). The second
file implements the model. This is the file you have to edit in this exercise.

\begin{itemize}
\item Consider the 1-dimensional advection model:
   \begin{eqnarray}
      \frac{\partial c}{\partial t}=v \frac{\partial c}{\partial x}
   \end{eqnarray}
   where $c$ typically describes the density of the particle being studied and
   $u$ is the velocity. On the left boundary $c$ is specified as
   $c_b(t)=1+\frac{1}{2}\sin(5 \pi t)$. Discretize this model on the interval
   $x=[0..1]$ with velocity $v=1$ using a 1st order upwind scheme on a grid of
   51 points. The time step is chosen such that the courant number $\frac{v
     \Delta t}{\Delta x}$ is approximately 1.
\item The deterministic model is extended into a stochastic model by adding a
  noise parameter $\omega$ on the left boundary. We use an AR(1) model to
  describe the noise.
\item Code your model in {\tt AdvectionStochModelInstance.java}. For
  inspiration, you will find in the same directory an implementation of the
  Lorenz model.
\item You can compile your model by typing "ant build" in the directory \\ {\tt
  exercise\_6/simple\_advection\_model}. This will create the file \\ {\tt
  bin/simple\_advection\_model.jar}.
\item Run the model. You can use file {\tt advectionSimulation.oda}. In order
  to be able to run your model, java must be able to find the file {\tt
    simple\_advection\_model.jar}. To accomplish this, you can copy your
  advection model jar-file to the bin-directory of your OpenDA installation or
  add the full path of your jar-file ({\tt simple\_advection\_model.jar}) to
  the java class path variable {\tt CLASSPATH}.\\ By default, the Windows
  scripts {\tt oda\_run\_gui.bat} and {\tt oda\_run\_batch.bat} use the JRE
  environment that is provided with OpenDA. If this JRE is incompatible with
  your JDK installation (Error: Unsupported major.minor version 51.0), use {\tt
    oda\_run\_batch.bat <inputfile> -jre "location of your JDK"} to overrule
  the default JRE.
\item Use the script {\tt plot\_movie.m} to visualize the model results. You
  will see that the model suffers from numerical diffusion. You can solve this
  by using a second order upwind method but this is not necessary for this
  exercise.
\item Create an ensemble of model simulations and study the model uncertainty
  in space and time.
\item The provided observation file {\tt observations.cvs} does not contain
  observations that correspond to the advection model. Create your own
  observation file for the locations $x=0.2$, $x=0.5$ and $x=0.7$. Use the
  model to create the values. Run the model with noise on the left boundary.
  Optionally generate some additional noise and add it to the generated
  observations. You can use the script {\tt generate\_obs} to simplify the
  creation of the observations file.
\item Using your generated observations, setup an ensemble Kalman filter run.
  Experiment with various numbers of ensembles, different settings of the AR(1)
  model.
\item Experiment with different intervals between the available observations.
  What do you observe. Is this behavior different from the Lorenz model?
  Evaluate the uncertainty of the estimates.
\item Experiment with only assimilating the data from one of the three
  locations and use the other locations as validation. What do you observe. Do
  all the locations have the same impact? Explain the behavior you observe.
\item Use the same data as generated before and use the Kalman filter now with
  different values of the system noise covariance and measurement noise
  covariance. Explain the behaviour that you observe.
\end{itemize}



\section{Exercise 7: Localization}
{\bf Directory: {\tt exercise\_localization}}

In this exercise, you will learn about localization techniques and how to use them in OpenDA. This exercise is inspired by the example model and experiments from "Impacts of localisation in the EnKF and EnOI: experiments with a small model", Peter R. Oke, Pavel Sakov and Stuart P. Corney, Ocean Dynamics (2007) 57: 32-45.

The model we use is a simple circular advection model 
\begin{equation}
\frac{\partial a}{\partial t}+u\frac{\partial a}{\partial x}=0,
\end{equation}
where $u=1$ is the speed of advection, $a$ is a model variable, $t$ is time and $x$ is a space ranging from 1 to 1000 with grid spacings of 1. The computational domain is periodic in $x$.

In this model, there are two related variables $a$ and $b$, where $b$ is initialized with a balance relationship:
\begin{equation}\label{eg:b_relation}
b= 0.5 + 10 \frac{da}{dx}
\end{equation}
and propagated with an advection model similar to the one for a, i.e.:
\begin{equation}
\frac{\partial b}{\partial t}+u\frac{\partial b}{\partial x}=0.
\end{equation}
Since $a$ and $b$ are propagated with the same flow, the balance relationship will remain valid also for $t>0$.
The relationship between $a$ and $b$ is motivated by the geostrophic balance relationship between pressure ($a$) and velocity ($b$) in oceanographic and atmospheric applications. 

In this experiment, we will only observe and assimilate $a$ and investigate how both $a$ and $b$ are updated. 
The ensemble is carefully constructed in order to have the correct statistics. The initial ensembles are generated offline and they will be read when the model is initialized in OpenDA. 

\begin{itemize}
\item Investigate the script {\tt generate\_ensemble.py} and figure out how the ensembles are generated.
\item Run Python script {\tt generate\_ensemble.py} to generate ensembles, observations, and true state for a 25, 50, and 100 ensemble experiment.  
\item Run the experiment for 50 ensemble members ({\tt enkf\_50.oda}).
\item The variables $a$ and $b$ can be compared to the true state using the Python script {\tt plot\_results.py}.
\item Run the experiment for 25 ensembles, copy the script {\tt plot\_results.py} to e.g. {\tt plot\_results\_25.py} and adjust it in order to read the results from {\tt enkf25\_results.py} (change 2nd line of the {\tt plot\_results.py} script). You will see that the 25 ensemble run is not able to improve the model.
\item Create input to run a 100-ensemble experiment. Note: do not forget to change the name of the output file (section {\tt resultWriter}) 
to avoid your previously-generated results being overwritten. 
\item Run an experiment with 25 ensembles with localization (see the script {\tt enkf\_25\_loc.oda}) and generate the plots.
\item The results (for 25 ensembles) with localization should look better than the experiment without localization.
\item Investigate whether the relation between $a$ and $b$ is violated by the various experiments. You can use the script {\tt check\_balance.py}.
\item Try changing the localization radius (initial value is 50) and see how the performance of the algorithms changes (both for results as balance between $a$ and $b$). You can plot the localization weight functions for each observation location ({\tt rho\_0}, {\tt rho\_1}, {\tt rho\_2}, and {\tt rho\_3}) as well.
\end{itemize}









\end{document}
