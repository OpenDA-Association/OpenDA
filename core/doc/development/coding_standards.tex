\svnidlong
{$HeadURL: $}
{$LastChangedDate: $}
{$LastChangedRevision: $}
{$LastChangedBy: $}

\odachapter{Coding standards}

\begin{tabular}{p{4cm}l}
\textbf{Origin:} \\
\textbf{Last update:}    & \svnfilemonth-\svnfileyear\\
\end{tabular}

\section{Introduction}
Good coding is partially a matter of taste. For a large system like \oda which is developed by various programmers we need some consensus on coding issues. At the moment we do not have a agreed on a coding standard for \oda. However we will state a number of rules in this chapter on which the developers have had discussions in the past.

\section{Use of }
Good coding is partially a matter of taste. For a large system like \oda which is developed by various programmers we need some consensus on coding issues. At the moment we do not have a agreed on a coding standard for \oda. However we will state a number of rules in this chapter on which the developers have had discussions in the past.

In \oda we do not use  for methods which implement an interface and do not override any "real" implementation. 

After developing \oda for a couple of years we noticed that the use of  in de code was very arbitrary. To uniform this we had to add @Overide in many locations or remove it. Modern development environments show the programmer whether a method overrides another implementation or only implements an interface. Manually adding @Override to the code does is therefore considered to be programming overhead which does not add insight in the code.


