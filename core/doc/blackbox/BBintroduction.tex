\svnidlong
{$HeadURL: $}
{$LastChangedDate: $}
{$LastChangedRevision: $}
{$LastChangedBy: $}

\odachapter{Using the Black Box wrapper}

\begin{tabular}{p{4cm}l}
\textbf{Contributed by:} & vacancy\\
\textbf{Last update:}    & \svnfilemonth-\svnfileyear\\
\end{tabular}

\section{Te schrijven hoofdstuk over de Black Box wrapper.}

Een algemene beschrijving van de concepten van de black box builder,
  deels al beschikbaar in het cursus materiaal.
\begin{itemize}
\item Hoe ziet het er uit? De filosofie achter de concepten.
\item Welke drie lagen zien we vaak? Koppeling aan data-assimilatie
  terminologie.
\item Welke bestanden zijn er?
\begin{itemize}
\item Welke Java interfaces horen bij welke bovengenoemde laag.
\item Welke standaard implementaties van de interfaces zijn beschikbaar. (en
  wanneer gebruik je ze? o.a. ExchangeItems (timeseries of gewoon een enkel
  veld), Observers (werkt alleen voor timeseries met 1 waarde per tijdstip;
  geen mogelijkheid tot ExchangeItem filtering in xml-configuratie)
\item Welke bestanden zijn actueel, obsolete, deels-doorgevoerd?
\item Verwijzing naar de voorbeeld xml-configuraties (welke up-to-date moeten
  worden).
\end{itemize}
\item Werken met exchangeItems, wat kunnen we voor knappe dingen met selecties
  maken (onmogelijk voor Observers) e.d. zodat je de noise op je model kan
  plakken. (Opm: is overigens slecht ondersteund, hier moet in \oda nog wat
  aan gebeuren!)
\end{itemize}
