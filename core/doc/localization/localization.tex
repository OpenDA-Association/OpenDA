\svnidlong
{$HeadURL: $}
{$LastChangedDate: 2014-07-08 18:10:04 +0200 (Tue, 08 Jul 2014) $}
{$LastChangedRevision: 4511 $}
{$LastChangedBy: vrielin $}

\odachapter{\oda Localization}

\begin{tabular}{p{4cm}l}
\textbf{Contributed by:} & Nils van Velzen, \vortech\\
\textbf{Last update:}    & \svnfilemonth-\svnfileyear\\
\end{tabular}

\section{Theory}
TO DO. A starting point is the paper: \oda-NEMO framework for ocean data assimilation, Ocean Dynamics (DOI 10.1007/s10236-016-0945-z).


\section{Configuration}
When model support distance based localization you can use it by adding the folowing option to your ENKF configuration file.
\begin{verbatim}
   <localization>hamill</localization>
   <distance>100</distance>
\end{verbatim}
The unit of the localization distance is not defined. The value is passed to the model and is model specific.

Automatic localization algorithm as poposed by  Yanfen Zhang and Dean S. Oliver. Evaluation and error analysis: Kalman gain regularisation versus covariance regularisation, Comput. Geosci (2011) 15:489-508, DOI 10.1007/s10596-010-9218-y
can be selected by setting

\begin{verbatim}
   <localization>zhang</localization>
\end{verbatim}

The bootstrap size is currently hard coded and connot be set in the configuration. The corresponing value can be found and set in the file AutoLocalizationZhang2011.java.





