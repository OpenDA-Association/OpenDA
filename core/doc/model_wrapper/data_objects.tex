\odachapter{Important data objects}

This chapter will describe some important data objects that can be used for multiple different wrappers when reading observations, model results or even updating states or boundary conditions.

\section{NetcdfDataObject}

The netcdf data object is very useful since many different model software packages use netcdf files for storing and reading data.

In \verb|org.openda.exchange.dataobjects.NetcdfDataObject|, the \oda implementation of netcdf data objects can be found.
It supports many different variable formats like scalar and gridded but not all (there can be an infinite number of different variable formats with infinite dimensions).
The most important ones will be described here.

\subsection{Scalar data}

The simplest format of scalar data has two dimensions, one for time and one for locations.
This will create exchange items for each location in each variable with the variable name combined with the location name as id.

\subsection{Gridded data}
There are two formats of gridded data variables supported:
\begin{itemize}
\item one with three dimensions containing time, $x$ and $y$: in this case 1 exchange item will be created for the whole variable. 
\item one with four  dimensions containing time, $x$, $y$ and an ensemble/realization dimension: in this case an exchange item for each member will be created.
\end{itemize}

\subsection{Extra arguments}

\subsubsection{Boolean: Lazy reading and writing}
It is possible to specify extra arguments for the data object. The first two are boolean values for lazy reading and lazy writing.
When lazy reading is set to true, the data will only be read from the netcdf file when the data is needed instead of when initializing the data object.
When lazy writing is set to true, the data will be written when the data object is closed instead of directly when the data is being changed by OpenDA.

\subsubsection{Key-value pair: \texttt{requiredExchangeItemId}}
In order to prevent a long list of arguments with a specific order, key-value pair arguments have been introduced.
Any number of \verb|requiredExchangeItemId=<ID>| arguments can be supplied which limits the exchange items being created to the ones supplied.
This can save a lot of memory and performance for large netcdf files.

\subsubsection{Key-value pair: \texttt{layerDimensionName}}
For scalar data which has multiple layers a \verb|layerDimensionName=<name>| needs to be provided. 
The size of that dimension will determine the amount of layer-specific exchange items that will be created.
When this argument is provided the id of the exchange items will be \verb|<variableName>.<locationName>.<layerIndex>|.

\subsubsection{Key-value pair: \texttt{allowTimeIndependentItems}}
The argument \verb|allowTimeIndependentItems=true/false| determines whether time-independent exchange items should be created.
Time-independent exchange items can be created for variables that do not depend on a time dimension.
Default is false.

